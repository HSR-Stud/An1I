\section{Funktionen}

\subsection{Gerade, ungerade und periodische Funktion}

Die Funktion $f$ heisst

\begin{itemize}
	\item \textit{gerade}, wenn \[\forall x \in DB(f): f(-x) = f(x)\]
	\item \textit{ungerade}, wenn \[\forall x \in DB(f): f(-x) = -f(x)\]
	\item \textit{periodisch} mit der Periode $p$, wenn \[\forall x \in DB(f): f(x + p) = f(x)\]
\end{itemize}

Die kleinste positive Periode einer periodischen Funktion heisst \textit{primitive Periode}.


\subsection{Umkehrbarkeit}

Die Funktion $f$ heisst \textit{umkehrbar}, wenn
%
\begin{displaymath}
	f(x_1) = f(x_2) \Rightarrow x_1 = x_2
\end{displaymath}

\subsection{Allgemeine Gleichungsregel}

Für jede umkehrbare Funktion $f$ gilt: Man darf beidseitig einer Funktion dieselbe umkehrbare
Funktion anwenden, wenn beide Seiten in ihrem Definitionsbereich liegen. Mathematisch ausgedrückt:
%
\begin{displaymath}
	\forall x_1, x_2 \in DB(f): x_1 = x_2 \Leftrightarrow f(x_1) = f(x_2)
\end{displaymath}


\subsection{Gleichungsregel für das Wegschaffen von Wurzeln}

Um die Wurzel auf der linken Seite der Gleichung $\sqrt{R} = S$ wegzuschaffen, sind zwei Fälle zu unterscheiden:

\begin{itemize}
	\item Wenn $S \geq 0$ ist, so ist die Gleichung äquivalent zu $R = S^2$
	\item Wenn $S < 0$ ist, ist die Gleichung unerfüllbar.
\end{itemize}

oder auf eine kurze Formel gebracht:
\begin{displaymath}
	\sqrt{R} = S \Leftrightarrow R = S^2 \wedge S \geq 0
\end{displaymath}


\subsection{Monotone Funktionen}

\begin{itemize}
	\item Sei $f$ eine monoton steigende Funktion. Dann gilt
		\begin{displaymath}
			f(x_1) < f(x_2) \Rightarrow x_1 < x_2
		\end{displaymath}

	\item Ist aber $f$ eine monoton fallende Funktion, so gilt
		\begin{displaymath}
			f(x_1) < f(x_2) \Rightarrow x_1 > x_2
		\end{displaymath}
\end{itemize}


\subsection{Allgemeine Ungleichungsregel}

Für jede streng monoton steigende Funktion $f$ gilt: Man darf beidseitig einer Ungleichung dieselbe
streng monoton steigende Funktion anwenden, wenn beide Seiten in ihrem Definitionsbereich liegen.
Oder mathematisch ausgedrückt:
%
\begin{displaymath}
	\forall x_1, x_2 \in DB(f): x_1 < x_2 \Leftrightarrow f(x_1) < f(x_2)
\end{displaymath}

Ferner gilt für jede streng monoton fallende Funktion $f$: Man darf beidseitig einer Ungleichung
dieselbe streng monoton fallende Funktion anwenden, wenn beide Seiten in ihrem Definitionsbereich
liegen. Dabei ist aber des Vergleichszeichen umzudrehen. Mathematisch ausgedrückt:
%
\begin{displaymath}
	\forall x_1, x_2 \in DB(f): x_1 < x_2 \Leftrightarrow f(x_1) > f(x_2)
\end{displaymath}


\subsection{Verkettung oder Komposition}

Gegeben seien die Funktionen $f$ und $g$. Dann nennt man die Funktion
%
\begin{displaymath}
	x \mapsto f(g(x))
\end{displaymath}
%
die \textit{Verkettung} oder \textit{Komposition} der Funktionen $f$ und $g$. Man bezeichnet sie mit
%
\begin{displaymath}
	f \circ g
\end{displaymath}
%
und liest das als \textit{$f$ nach $g$}.


\subsection{Graphen der Verkettung von Funktionen mit linearen Funktionen}

Der Graph der Funktion $f$ sei bekannt. Dann geht der Graph der Funktion
%
\begin{displaymath}
	x \mapsto af(x) + b
\end{displaymath}
%
aus jenem von f durch folgende geometrische Operationen hervor (Reihenfolge wesentlich!)
\begin{enumerate}
	\item Vertikale Skalierung um den Faktor $|a|$
	\begin{itemize}
		\item Wenn $a < 0$ zusätzlich eine Spiegelung an der 1. Koordinatenachse
	\end{itemize}
	\item Vertikalverschiebung um $|b|$ und zwar
	\begin{itemize}
		\item Nach oben, wenn $b > 0$
		\item Nach unten, wenn $b < 0$
	\end{itemize}
\end{enumerate}

Ferner geht der Graph der Funktion
%
\begin{displaymath}
	x \mapsto f(ax + b)
\end{displaymath}
%
aus jenem $f$ durch folgende geometrische Operationen hervor (Reihenfolge wesentlich!)
\begin{enumerate}
	\item Horizontalverschiebung um $|b|$ und zwar
	\begin{itemize}
		\item Nach links, wenn $b > 0$
		\item Nach rechts, wenn $b < 0$
	\end{itemize}
	\item Horizontale Skalierung um den Faktor $\displaystyle\frac{1}{|a|}$
	\begin{itemize}
		\item Wenn $a < 0$ zusätzlich eine Spiegelung an der 2. Koordinatenachse	
	\end{itemize}
\end{enumerate}


\subsection{Umkehrfunktion}

Sei $f$ eine umkehrbare Funktion. Dann heisst die Funktion $f^{-1}$, für welche gilt
%
\begin{displaymath}
	f^{-1}(y) = x \Leftrightarrow y = f(x)
\end{displaymath}
%
die \textit{Umkehrfunktion} von $f$. Für termdefinierte Funktionen gilt also
%
\begin{displaymath}
	f = x \mapsto y \Leftrightarrow f^{-1} = y \mapsto x
\end{displaymath}
%
In anderen Worten: Bei der Umkehrfunktion werden einfach die Rollen von Argument und Funktionswert
vertauscht. Dies läuft auf eine Spiegelung des Graphen der gegebenen Funktion an der ersten
Quadrantenhalbierenden hinaus.


\subsection{Graphen von Umkehrfunktionen}

Sei $f$ eine umkehrbare Funktion. Dann ist der Graph von $f^{-1}$ das Spiegelbild des Graphen von
$f$ an der 1. Quadrantenhalbierenden.


\subsection{Verkettung einer Funktion mit ihrer Umkehrfunktion}

Sei $f$ eine umkehrbare Funktion. Dann gilt
%
\begin{displaymath}
	\forall x \in DB(f): f^{-1}(f(x)) = x
\end{displaymath}
%
oder knapper
%
\begin{displaymath}
	f^{-1} \circ f = id_{DB(f)}
\end{displaymath}

\subsection{Eigentliche und Uneigentliche Grenzwerte}

Eigentliche Grenzwerte sind Grenzwerte, welche gegen eine reelle Zahl streben. Uneigentliche
Grenzwerte sind Grenzwerte, welche gegen Unendlich (positiv oder negativ) streben.

\subsection{Stetigkeit}

Wenn die reelle Funktion $f$ an der Stelle $a$ definiert ist und
%
\begin{displaymath}
	\lim_{x \to a+} f(x) = \lim_{x \to a-} f(x) = f(a)
\end{displaymath}
%
gilt, dann heisst die Funktion bei $a$ stetig.

Vereinfacht gesagt, kann man sagen, dass eine stetige Funktion gezeichnet werden kann, ohne den
Stift abzusetzen.
